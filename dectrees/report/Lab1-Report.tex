% Very simple template for lab reports. Most common packages are already included.
\documentclass[a4paper, 11pt]{article}
\usepackage[utf8]{inputenc} % Change according to your file encoding
\usepackage{graphicx}
\usepackage{url}
\usepackage{listings}
\usepackage{xcolor}
\usepackage{float} % provides [H] placement to pin floats
\usepackage{placeins} % provides \FloatBarrier to prevent floats from passing barriers


%opening
\title{Report 1: Dectrees}
\author{Group XXX: XXX, Riccardo Fragale}
\date{\today{}}

\begin{document}

\maketitle

\section{Assignment 0}
\textit{Each one of the datasets has properties which makes them hard to learn.
Motivate which of the three problems is most difficult for a decision
tree algorithm to learn.}

The bigger challenge for all the three datasets is the number of training samples compared to test samples.
Let's consider tha dataset \textbf{MONK1} as example to justify that: it has exactly 431 test samples 
while the number of training samples is 123. Considering that we don't have a validation set and 
we don't use a k-fold cross-validation, we have a really limited amount of data to train the model on.
This reasoning is valid also for \textbf{MONK2} and \textbf{MONK3}.
This makes the decision-tree less able to generalise and have a complete picture of the classification problem to be solved.

Going in detail to the single datasets we can say that:
\begin{itemize}
    \item \textbf{MONK1} appears to be the less hard of the three to model as there is a clearer rule that depends only on 3 attributes 
    while the others are mostly irrelevant to predict the classification as true or false
    \item \textbf{MONK2} is the harder to classify because the pattern to be identified is more complex as discrete attributes have value 1 that is not repeated but simply two of 
    them, randomically, get the value 1
    \item \textbf{MONK3} has the lower number of training samples so it is harder for the model to get an accurate classification.
    Moreover there is a 5\% additional classification noise in the training set which makes the work even harder.
\end{itemize}



\section{Assignment 2}
\textit{The file dtree.py defines a function entropy which
calculates the entropy of a dataset. Import this file along with the
monks datasets and use it to calculate the entropy of the training
datasets.}

\begin{center}
  \begin{tabular*}{0.9\textwidth}{|c|c@{\extracolsep{\fill}}c|}
    \hline
    Dataset & Entropy & \\
    \hline\hline
    MONK-1 & 1.0 & \\
    \hline
    MONK-2 & 0.957117428264771 & \\
    \hline
    MONK-3 & 0.9998061328047111& \\
    \hline
  \end{tabular*}
\end{center}

\section{Assignment 2}
\textit{Explain entropy for a uniform distribution and a
non-uniform distribution, present some example distributions with
high and low entropy.}
When we tak about a uniform distribution, we are describing a situation where every possible outcome has the same 
probability to appear in a single sample. This implies bigger unpredictability.
A good example of it is the case of a fair coin flip where each of the 2 outcomes has 1/2 of probability.
This is the maximum level of entropy possible for a set of outcomes as no outcome is more likely than any other.
 INSERIRE FOTO
Then we have non-uniform distributions where some outcomes are more likely than others. In this case we 
have lower unpredictability and less surprise when certain outcomes appear.
As a consequence the entropy is lower than a uniform distribution case.
If the distribution is highly unbalanced towards a certain outcome the entropy might be very very low.
An example could be the case of a very unfair coin where the outcome HEAD appears 99\% of the times.
 INSERIRE FOTO




\section{Assignment 3}
\textit{Use the function averageGain (defined in dtree.py)
to calculate the expected information gain corresponding to each of
the six attributes. Note that the attributes are represented as in-
stances of the class Attribute (defined in monkdata.py) which you
can access via m.attributes[0], ..., m.attributes[5]. Based on
the results, which attribute should be used for splitting the examples
at the root node?}


\section{Assignment 4}
\textit{For splitting we choose the attribute that maximizes
the information gain, Eq.3. Looking at Eq.3 how does the entropy of
the subsets, $S_k$, look like when the information gain is maximized?
How can we motivate using the information gain as a heuristic for
picking an attribute for splitting? Think about reduction in entropy
after the split and what the entropy implies.}


\section{Assignment 5}
Build the full decision trees for all three Monk datasets using
\texttt{buildTree}.  Then, use the function \texttt{check} to measure the performance
of the decision tree on both the training and test datasets.

For example to built a tree for \texttt{monk1} and compute the performance on the test data
you could use
\begin{verbatim}
import monkdata as m
import dtree as d

t=d.buildTree(m.monk1, m.attributes);
print(d.check(t, m.monk1test))
\end{verbatim}

Compute the train and test set errors for the three Monk datasets for
the full trees. Were your assumptions about the datasets correct? Explain the 
results you get for the training and test datasets.


\section{Assignment 6}
\textit{Explain pruning from a bias variance trade-off perspective.}

\section{Assignment 7}
Evaluate the effect pruning has on the test
error for the \texttt{monk1} and \texttt{monk3} datasets, in
particular determine the optimal partition into training and pruning
by optimizing the parameter \texttt{fraction}.  Plot the
classification error on the test sets as a function of the parameter
\texttt{fraction} $\in \{0.3,0.4,0.5,0.6,0.7,0.8\}$. 



\end{document}