% Very simple template for lab reports. Most common packages are already included.
\documentclass[a4paper, 11pt]{article}
\usepackage[utf8]{inputenc} % Change according to your file encoding
\usepackage{graphicx}
\usepackage{url}
\usepackage{listings}
\usepackage{xcolor}
\usepackage{float} % provides [H] placement to pin floats
\usepackage{placeins} % provides \FloatBarrier to prevent floats from passing barriers


%opening
\title{Report 1: Dectrees}
\author{Group XXX: XXX, Riccardo Fragale}
\date{\today{}}

\begin{document}

\maketitle

\section{Assignment 0}
\textit{Each one of the datasets has properties which makes them hard to learn.
Motivate which of the three problems is most difficult for a decision
tree algorithm to learn.}

The bigger challenge for all the three datasets is the number of training samples compared to test samples.
Let's consider tha dataset \textbf{MONK1} as example to justify that: it has exactly 431 test samples 
while the number of training samples is 123. Considering that we don't have a validation set and 
we don't use a k-fold cross-validation, we have a really limited amount of data to train the model on.
This reasoning is valid also for \textbf{MONK2} and \textbf{MONK3}.
This makes the decision-tree less able to generalise and have a complete picture of the classification problem to be solved.

Going in detail to the single datasets we can say that:
\begin{itemize}
    \item \textbf{MONK1} appears to be the less hard of the three to model as there is a clearer rule that depends only on 3 attributes 
    while the others are mostly irrelevant to predict the classification as true or false
    \item \textbf{MONK2} is the harder to classify because the pattern to be identified is more complex as discrete attributes have value 1 that is not repeated but simply two of 
    them, randomically, get the value 1
    \item \textbf{MONK3} has the lower number of training samples so it is harder for the model to get an accurate classification.
    Moreover there is a 5\% additional classification noise in the training set which makes the work even harder.
\end{itemize}



\section{Assignment 2}
\textit{The file dtree.py defines a function entropy which
calculates the entropy of a dataset. Import this file along with the
monks datasets and use it to calculate the entropy of the training
datasets.}

\begin{center}
  \begin{tabular*}{0.9\textwidth}{|c|c@{\extracolsep{\fill}}c|}
    \hline
    Dataset & Entropy & \\
    \hline\hline
    MONK-1 & 1.0 & \\
    \hline
    MONK-2 & 0.957117428264771 & \\
    \hline
    MONK-3 & 0.9998061328047111& \\
    \hline
  \end{tabular*}
\end{center}

\section{Assignment 2}
\textit{Explain entropy for a uniform distribution and a
non-uniform distribution, present some example distributions with
high and low entropy.}
When we tak about a uniform distribution, we are describing a situation where every possible outcome has the same 
probability to appear in a single sample. This implies bigger unpredictability.
A good example of it is the case of a fair coin flip where each of the 2 outcomes has 1/2 of probability.
This is the maximum level of entropy possible for a set of outcomes as no outcome is more likely than any other.
 INSERIRE FOTO
Then we have non-uniform distributions where some outcomes are more likely than others. In this case we 
have lower unpredictability and less surprise when certain outcomes appear.
As a consequence the entropy is lower than a uniform distribution case.
If the distribution is highly unbalanced towards a certain outcome the entropy might be very very low.
An example could be the case of a very unfair coin where the outcome HEAD appears 99\% of the times.
 INSERIRE FOTO




\section{Assignment 3}
\textit{Use the function averageGain (defined in dtree.py)
to calculate the expected information gain corresponding to each of
the six attributes. Note that the attributes are represented as in-
stances of the class Attribute (defined in monkdata.py) which you
can access via m.attributes[0], ..., m.attributes[5]. Based on
the results, which attribute should be used for splitting the examples
at the root node?}

Just for the sake of clarity, this is the equation corresponding to the gain we just talked about.
\begin{equation}
\textrm{Gain}(S,A) = \textrm{Entropy}(S) -
 \sum_{k \in \textrm{values}(A)} \frac{|S_k|}{|S|} \textrm{Entropy}(S_k)
\end{equation}

\begin{center}
  Information Gain\\[0.5ex]
  \begin{tabular*}{\textwidth}{|c@{\extracolsep{\fill}}|c|c|c|c|c|c|}
    \hline
    Dataset & $a_1$ & $a_2$ & $a_3$ & $a_4$ & $a_5$ & $a_6$ \\
    \hline
    \verb!MONK-1 ! & 0.0753 & 0.0058 & 0.0047 & 0.0263 & 0.2870 & 0.0008 \\
    \hline
    \verb!MONK-2 ! & 0.0038 & 0.0025 & 0.0011 & 0.0157 & 0.0173 & 0.0062 \\
    \hline
    \verb!MONK-3 ! & 0.0071 & 0.2937 & 0.0008 & 0.0029 & 0.2559 & 0.0071 \\
    \hline
  \end{tabular*}
\end{center}

The attribute that carries the highest information gain is the one to be selected for the splitting of the dataset
in the most effective way. 
For \textit{MONK1} dataset, attribute $a_5$ carries an information gain of 0.2870 which is a lot more than any other attribute
so it's the one to be chosen. A similar approach leads us to select attribute $a_5$ again for \textit{MONK2} dataset even though here the difference
is far less than in the previous case. As we individuated correctly in \textit{Assignment 0} this dataset is the harder to classify and we have a further proof
as no single attribute has a very big information gain when known. 
Finally, for \textit{MONK3} dataset the one that has to be chosen to split the dataset is attribute $a_2$ with an information gain of 0.2937. In this case 
we have a big gain but there is another relecant attribute which is again $a_5$ but the gain is lower than $a_2$ by 0.04.


\section{Assignment 4}
\textit{For splitting we choose the attribute that maximizes
the information gain, Eq.1. Looking at Eq.1 how does the entropy of
the subsets, $S_k$, look like when the information gain is maximized?
How can we motivate using the information gain as a heuristic for
picking an attribute for splitting? Think about reduction in entropy
after the split and what the entropy implies.}

Since we want a high gain, and the value of the entropy of S is fixed, then necessarily maximizing the gain 
implies that the entropy of the subsets $S_k$ is minimized.
After the split, the entropy of the single subsets is very low while the entropy of the whole dataset remains the same.
As entropy is a measure of the unpredictability of a dataset, when we minimize it in the subsets (by choosing a splitting attribute) we are actually on the right path
towards identifying small groups of the data that have to be classified with the same value as in each single subset $S_k$ we have higher predictability than before the split.
Hence, we are optimizeing the complexity of the tree (and as a consequence of the learning procedure) by selecting the attribute with the maximum gain.


\section{Assignment 5}
Build the full decision trees for all three Monk datasets using
\texttt{buildTree}.  Then, use the function \texttt{check} to measure the performance
of the decision tree on both the training and test datasets.

For example to built a tree for \texttt{monk1} and compute the performance on the test data
you could use
\begin{verbatim}
import monkdata as m
import dtree as d

t=d.buildTree(m.monk1, m.attributes);
print(d.check(t, m.monk1test))
\end{verbatim}

Compute the train and test set errors for the three Monk datasets for
the full trees. Were your assumptions about the datasets correct? Explain the 
results you get for the training and test datasets.

The following are the results we obtained on the three datasets:
\begin{center}
  \begin{tabular*}{0.7\textwidth}{|c|@{\extracolsep{\fill}}c|c|}
    \hline
    & $E_\textrm{train}$ & $E_\textrm{test}$ \\
    \hline\hline
    \verb#MONK-1# & 0.0 & 0.171\\
    \hline
    \verb#MONK-2# & 0.0 & 0.308 \\
    \hline
    \verb#MONK-3# & 0.0 & 0.056 \\
    \hline
  \end{tabular*}
\end{center}

For all the three datasets we have a train error of exactly 0; this implies that all predictions are done correctly.
This is absolutely normal as the model was trained on those data so it knows all the necessary characteristics of them
and the value of each sample. Despite that, having it exactly 0 means that the algorithm may be overfitting 
the training dataset.
Regarding test error, we can say that the third tree has the best performances (95\% of predictions correct) while the second one
offers the worst performances. This is another proof that it's the dataset harder to classify due to the 
characteristics we mentioned in \textit{Assignment0}.


\section{Assignment 6}
\textit{Explain pruning from a bias variance trade-off perspective.}
The idea behind the decision tree model is that the algorithm grows without any constraint trying to split
the data until each leaf of the three is clearly and undoubtedly classifiable.
This process generally leads to a very very low bias ( since the model is able to distinguish with precision almost every case )
but at the same time a pretty high variance; since it is very precise, changing just a few samples of the training dataset implies
relevant changes in the entire structure of the tree. Moreover, we might have overspecialization on the training dataset 
and so lower ability to generalize and to be really effective while preicting on new samples.
Since we want to improve our models towards better specialization we decide to remove branches that provide little predicting value (pruning a set of "irrelevant" nodes)
acccepting an increase in terms of bias as the model is a bit too simple to capture every single detail but in 
exchange for a significant reduction of the variance (slight changes in the training dataset won't impact as much as before).

\section{Assignment 7}
Evaluate the effect pruning has on the test
error for the \texttt{monk1} and \texttt{monk3} datasets, in
particular determine the optimal partition into training and pruning
by optimizing the parameter \texttt{fraction}.  Plot the
classification error on the test sets as a function of the parameter
\texttt{fraction} $\in \{0.3,0.4,0.5,0.6,0.7,0.8\}$. 

The following is the image with the data obtained applying pruning procedure 
and determining the optimal partition fraction.

\begin{figure}[H]
  \centering
  \fbox{\includegraphics[width=0.9\textwidth]{../python/ass7lab1.png}}
  \caption{Test classification error as a function of the training/pruning fraction (MONK datasets, n = 500).}
  \label{fig:ass7lab1}
\end{figure}

We can derive from the graph that whatever fraction between 0.5 and 0.7 is okay for having the best model possible.
The ideal one is around 0.6 for both MONK1 and MONK3.
They offer a very limited error and a reduced variance. It is clear that we need a bit more data for the training rather than
from validation.





\end{document}