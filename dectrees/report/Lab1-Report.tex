% Very simple template for lab reports. Most common packages are already included.
\documentclass[a4paper, 11pt]{article}
\usepackage[utf8]{inputenc} % Change according to your file encoding
\usepackage{graphicx}
\usepackage{url}
\usepackage{listings}
\usepackage{xcolor}
\usepackage{float} % provides [H] placement to pin floats
\usepackage{placeins} % provides \FloatBarrier to prevent floats from passing barriers


%opening
\title{Report 1: Dectrees}
\author{Group XXX: XXX, Riccardo Fragale}
\date{\today{}}

\begin{document}

\maketitle

\section{Assignment 0}
\textit{Each one of the datasets has properties which makes them hard to learn.
Motivate which of the three problems is most difficult for a decision
tree algorithm to learn.}

The bigger challenge for all the three datasets is the number of training samples compared to test samples.
Let's consider tha dataset \textbf{MONK1} as example to justify that: it has exactly 431 test samples 
while the number of training samples is 123. Considering that we don't have a validation set and 
we don't use a k-fold cross-validation, we have a really limited amount of data to train the model on.
This reasoning is valid also for \textbf{MONK2} and \textbf{MONK3}.
This makes the decision-tree less able to generalise and have a complete picture of the classification problem to be solved.





\end{document}